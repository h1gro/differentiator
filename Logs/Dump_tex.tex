\documentclass[a4paper,12pt]{article}
\usepackage{cmap}
\usepackage[T2A]{fontenc}
\usepackage[utf8]{inputenc}
\usepackage[english,russian]{babel}
\usepackage{graphicx}
\graphicspath{{noiseimages/}}
\usepackage{geometry}
\geometry{papersize={25 cm, 25 cm}}
\geometry{left=2cm}
\geometry{right=2cm}
\geometry{top=2cm}
\geometry{bottom=2cm}\usepackage{enumitem}
\date{}


\newtheorem{task}{Задача}
\begin{document}
\begin{titlepage}
\begin{center}
	\textsc{MOSKOW INSTITUTE OF PHYSICS AND TECHNOLOGY}
	\vspace{2ex}

\end{center}
\vspace{10ex}
\begin{center}
	\vspace{24ex}

	\vspace{2ex}
	\textbf{\Large{Differentiator}}
	\vspace{34ex}
	\begin{flushright}
	\noindent
	Done by:
	\textit{Komarov Artem}
	\end{flushright}
	\vfill
	Dolgoprudny, 2024
\end{center}
\end{titlepage}
\newpage
\section{Expression}
\begin{center}
	\text{In this simple example - function, that can solve my little brother
	I'll show you how works my Differentiator.}
	\vspace{2ex}

	\textit{$f(x) = ( x  -  1 ) \cdot {e} ^ {\frac { 1 }{ 2  \cdot  x } } + {e} ^ {( 2  \cdot  x  -  1  - { 6 } ^ { x } + \ln(\ln(\ch({ 153 } ^ { x }) ) ) )} $}
\end{center}
\subsection{Cathedra of MATAN told}
\begin{center}
	\textit{$( x )' =  1 $}
\end{center}
\subsection{In Bashkiria kids know}
\begin{center}
	\textit{$( 1 )' =  0 $}
\end{center}
\subsection{This understand Australopithecus}
\begin{center}
	\textit{$( x  -  1 )' =  1  -  0 $}
\end{center}
\subsection{Extremly obvious}
\begin{center}
	\textit{$( 1 )' =  0 $}
\end{center}
\subsection{When I am sleeping this dreaming to me}
\begin{center}
	\textit{$( 2 )' =  0 $}
\end{center}
\subsection{If you did not know that I am so sorry for you}
\begin{center}
	\textit{$( x )' =  1 $}
\end{center}
\subsection{It's like 2 + 2}
\begin{center}
	\textit{$( 2  \cdot  x )' =  0  \cdot  x  +  2  \cdot  1 $}
\end{center}
\subsection{Be smart and learn it finally}
\begin{center}
	\textit{$(\frac { 1 }{ 2  \cdot  x } )' = \frac { 0  \cdot  2  \cdot  x  -  1  \cdot ( 0  \cdot  x  +  2  \cdot  1 )}{ 2  \cdot  x  \cdot  2  \cdot  x } $}
\end{center}
\subsection{Ilsaf is FIVT, they know about that}
\begin{center}
	\textit{$({e} ^ {\frac { 1 }{ 2  \cdot  x } })' = {e} ^ {\frac { 1 }{ 2  \cdot  x } } \cdot \frac { 0  \cdot  2  \cdot  x  -  1  \cdot ( 0  \cdot  x  +  2  \cdot  1 )}{ 2  \cdot  x  \cdot  2  \cdot  x } $}
\end{center}
\subsection{So that's easy I think, but not for you}
\begin{center}
	\textit{$(( x  -  1 ) \cdot {e} ^ {\frac { 1 }{ 2  \cdot  x } })' = ( 1  -  0 ) \cdot {e} ^ {\frac { 1 }{ 2  \cdot  x } } + ( x  -  1 ) \cdot {e} ^ {\frac { 1 }{ 2  \cdot  x } } \cdot \frac { 0  \cdot  2  \cdot  x  -  1  \cdot ( 0  \cdot  x  +  2  \cdot  1 )}{ 2  \cdot  x  \cdot  2  \cdot  x } $}
\end{center}
\subsection{Bolzano-Weierstrass theorem makes you think}
\begin{center}
	\textit{$( 2 )' =  0 $}
\end{center}
\subsection{My dog can solve it, come on man}
\begin{center}
	\textit{$( x )' =  1 $}
\end{center}
\subsection{Note that}
\begin{center}
	\textit{$( 2  \cdot  x )' =  0  \cdot  x  +  2  \cdot  1 $}
\end{center}
\subsection{In some intelectual grops of people it mean: Base}
\begin{center}
	\textit{$( 1 )' =  0 $}
\end{center}
\subsection{Shut the f*ck up and calculate it!}
\begin{center}
	\textit{$( 2  \cdot  x  -  1 )' =  0  \cdot  x  +  2  \cdot  1  -  0 $}
\end{center}
\subsection{Not your level, I guess}
\begin{center}
	\textit{$( x )' =  1 $}
\end{center}
\subsection{I solve my lovely-lovely crocodile!)}
\begin{center}
	\textit{$({ 6 } ^ { x })' = { 6 } ^ { x } \cdot \ln( 6 )  \cdot  1 $}
\end{center}
\subsection{Didn't even break a sweat}
\begin{center}
	\textit{$( 2  \cdot  x  -  1  - { 6 } ^ { x })' =  0  \cdot  x  +  2  \cdot  1  -  0  - { 6 } ^ { x } \cdot \ln( 6 )  \cdot  1 $}
\end{center}
\subsection{Who could I be if I hadn't do math?}
\begin{center}
	\textit{$( x )' =  1 $}
\end{center}
\subsection{What's the meaning of life?}
\begin{center}
	\textit{$({ 153 } ^ { x })' = { 153 } ^ { x } \cdot \ln( 153 )  \cdot  1 $}
\end{center}
\subsection{Who am I?}
\begin{center}
	\textit{$(\ch({ 153 } ^ { x }) )' = \sh({ 153 } ^ { x })  \cdot { 153 } ^ { x } \cdot \ln( 153 )  \cdot  1 $}
\end{center}
\subsection{Diffucult quastions, but easy derivative!)))}
\begin{center}
	\textit{$(\ln(\ch({ 153 } ^ { x }) ) )' = \frac { 1 }{\ch({ 153 } ^ { x }) }  \cdot \sh({ 153 } ^ { x })  \cdot { 153 } ^ { x } \cdot \ln( 153 )  \cdot  1 $}
\end{center}
\subsection{Yeeeh get it!}
\begin{center}
	\textit{$(\ln(\ln(\ch({ 153 } ^ { x }) ) ) )' = \frac { 1 }{\ln(\ch({ 153 } ^ { x }) ) }  \cdot \frac { 1 }{\ch({ 153 } ^ { x }) }  \cdot \sh({ 153 } ^ { x })  \cdot { 153 } ^ { x } \cdot \ln( 153 )  \cdot  1 $}
\end{center}
\subsection{Cathedra of MATAN told}
\begin{center}
	\textit{$( 2  \cdot  x  -  1  - { 6 } ^ { x } + \ln(\ln(\ch({ 153 } ^ { x }) ) ) )' =  0  \cdot  x  +  2  \cdot  1  -  0  - { 6 } ^ { x } \cdot \ln( 6 )  \cdot  1  + \frac { 1 }{\ln(\ch({ 153 } ^ { x }) ) }  \cdot \frac { 1 }{\ch({ 153 } ^ { x }) }  \cdot \sh({ 153 } ^ { x })  \cdot { 153 } ^ { x } \cdot \ln( 153 )  \cdot  1 $}
\end{center}
\subsection{In Bashkiria kids know}
\begin{center}
	\textit{$({e} ^ {( 2  \cdot  x  -  1  - { 6 } ^ { x } + \ln(\ln(\ch({ 153 } ^ { x }) ) ) )})' = {e} ^ {( 2  \cdot  x  -  1  - { 6 } ^ { x } + \ln(\ln(\ch({ 153 } ^ { x }) ) ) )} \cdot ( 0  \cdot  x  +  2  \cdot  1  -  0  - { 6 } ^ { x } \cdot \ln( 6 )  \cdot  1  + \frac { 1 }{\ln(\ch({ 153 } ^ { x }) ) }  \cdot \frac { 1 }{\ch({ 153 } ^ { x }) }  \cdot \sh({ 153 } ^ { x })  \cdot { 153 } ^ { x } \cdot \ln( 153 )  \cdot  1 )$}
\end{center}
\subsection{This understand Australopithecus}
\begin{center}
	\textit{$(( x  -  1 ) \cdot {e} ^ {\frac { 1 }{ 2  \cdot  x } } + {e} ^ {( 2  \cdot  x  -  1  - { 6 } ^ { x } + \ln(\ln(\ch({ 153 } ^ { x }) ) ) )})' = ( 1  -  0 ) \cdot {e} ^ {\frac { 1 }{ 2  \cdot  x } } + ( x  -  1 ) \cdot {e} ^ {\frac { 1 }{ 2  \cdot  x } } \cdot \frac { 0  \cdot  2  \cdot  x  -  1  \cdot ( 0  \cdot  x  +  2  \cdot  1 )}{ 2  \cdot  x  \cdot  2  \cdot  x }  + {e} ^ {( 2  \cdot  x  -  1  - { 6 } ^ { x } + \ln(\ln(\ch({ 153 } ^ { x }) ) ) )} \cdot ( 0  \cdot  x  +  2  \cdot  1  -  0  - { 6 } ^ { x } \cdot \ln( 6 )  \cdot  1  + \frac { 1 }{\ln(\ch({ 153 } ^ { x }) ) }  \cdot \frac { 1 }{\ch({ 153 } ^ { x }) }  \cdot \sh({ 153 } ^ { x })  \cdot { 153 } ^ { x } \cdot \ln( 153 )  \cdot  1 )$}
\end{center}
\subsection{Extremly obvious}
\begin{center}
	\textit{$(( x  -  1 ) \cdot {e} ^ {\frac { 1 }{ 2  \cdot  x } } + {e} ^ {( 2  \cdot  x  -  1  - { 6 } ^ { x } + \ln(\ln(\ch({ 153 } ^ { x }) ) ) )})' = ( 1  -  0 ) \cdot {e} ^ {\frac { 1 }{ 2  \cdot  x } } + ( x  -  1 ) \cdot {e} ^ {\frac { 1 }{ 2  \cdot  x } } \cdot \frac { 0  \cdot  2  \cdot  x  -  1  \cdot ( 0  \cdot  x  +  2  \cdot  1 )}{ 2  \cdot  x  \cdot  2  \cdot  x }  + {e} ^ {( 2  \cdot  x  -  1  - { 6 } ^ { x } + \ln(\ln(\ch({ 153 } ^ { x }) ) ) )} \cdot ( 0  \cdot  x  +  2  \cdot  1  -  0  - { 6 } ^ { x } \cdot \ln( 6 )  \cdot  1  + \frac { 1 }{\ln(\ch({ 153 } ^ { x }) ) }  \cdot \frac { 1 }{\ch({ 153 } ^ { x }) }  \cdot \sh({ 153 } ^ { x })  \cdot { 153 } ^ { x } \cdot \ln( 153 )  \cdot  1 )$}
\end{center}
\subsection{Final Derivative}
\begin{center}
	\textit{$(( x  -  1 ) \cdot {e} ^ {\frac { 1 }{ 2  \cdot  x } } + {e} ^ {( 2  \cdot  x  -  1  - { 6 } ^ { x } + \ln(\ln(\ch({ 153 } ^ { x }) ) ) )})' = {e} ^ {\frac { 1 }{ 2  \cdot  x } } + ( x  -  1 ) \cdot {e} ^ {\frac { 1 }{ 2  \cdot  x } } \cdot \frac { -2 }{ 2  \cdot  x  \cdot  2  \cdot  x }  + {e} ^ {( 2  \cdot  x  -  1  - { 6 } ^ { x } + \ln(\ln(\ch({ 153 } ^ { x }) ) ) )} \cdot ( 2  - { 6 } ^ { x } \cdot \ln( 6 )  + \frac { 1 }{\ln(\ch({ 153 } ^ { x }) ) }  \cdot \frac { 1 }{\ch({ 153 } ^ { x }) }  \cdot \sh({ 153 } ^ { x })  \cdot { 153 } ^ { x } \cdot \ln( 153 ) )$}
\end{center}
\section{Result}
\begin{center}
	\textit{$f(x) = ( x  -  1 ) \cdot {e} ^ {\frac { 1 }{ 2  \cdot  x } } + {e} ^ {( 2  \cdot  x  -  1  - { 6 } ^ { x } + \ln(\ln(\ch({ 153 } ^ { x }) ) ) )}$}
	\vspace{6ex}

	\textit{$f'(x) = {e} ^ {\frac { 1 }{ 2  \cdot  x } } + ( x  -  1 ) \cdot {e} ^ {\frac { 1 }{ 2  \cdot  x } } \cdot \frac { -2 }{ 2  \cdot  x  \cdot  2  \cdot  x }  + {e} ^ {( 2  \cdot  x  -  1  - { 6 } ^ { x } + \ln(\ln(\ch({ 153 } ^ { x }) ) ) )} \cdot ( 2  - { 6 } ^ { x } \cdot \ln( 6 )  + \frac { 1 }{\ln(\ch({ 153 } ^ { x }) ) }  \cdot \frac { 1 }{\ch({ 153 } ^ { x }) }  \cdot \sh({ 153 } ^ { x })  \cdot { 153 } ^ { x } \cdot \ln( 153 ) )$}
\end{center}
\begin{center}
	\vspace{12ex}
	\textbf{\Large{DERIVATIVE IS KILLED!!!}}

	\vspace{2ex}
	\textbf{Thank you for your attention!}

	\vspace{2ex}
	\textbf{BOTAITE!}
\end{center}
\end{document}
